% Adaptado para IFSP-BRA por Profa. Dra. Ana Paula Müller Giancoli - 27/05/2023

% Não modificar - inicio
% -- Classe Documento
\documentclass[
	% -- opções da classe memoir --
	hidelinks,
	12pt,				% tamanho da fonte
	openright,			% capítulos começam em pág ímpar (insere página vazia caso preciso)
	oneside,			% para impressão em verso e anverso. Oposto a oneside, twoside
	a4paper,			% tamanho do papel. 
	%normalfigtabnum,
	%pnumromarab,
	% -- opções da classe abntex2 --
	%chapter=TITLE,		% títulos de capítulos convertidos em letras maiúsculas
	%section=TITLE,		% títulos de seções convertidos em letras maiúsculas
	%subsection=TITLE,	% títulos de subseções convertidos em letras maiúsculas
	%subsubsection=TITLE,% títulos de subsubseções convertidos em letras maiúsculas
	% -- opções do pacote babel --
	english,			% idioma adicional para hifenização
	french,				% idioma adicional para hifenização
	spanish,			% idioma adicional para hifenização
	brazil,				% o último idioma é o principal do documento
]{abntex2}

\usepackage{array}
\newcolumntype{P}[1]{>{\centering\arraybackslash}p{#1}}
% --
\usepackage[document]{ragged2e}
% ---
% Pacotes básicos 
% ---
% Usa a fonte Helvet que é parecida com Arial
\usepackage{helvet} 
% define como default.
\renewcommand{\familydefault}{\sfdefault} 
% Selecao de codigos de fonte.
\usepackage[T1]{fontenc}	
% Codificacao do documento (conversão automática dos acentos)
\usepackage[utf8]{inputenc}	
% Usado pela Ficha catalográfica
\usepackage{lastpage}	
% Indenta o primeiro parágrafo de cada seção.
\usepackage{indentfirst}		
\usepackage{graphicx}
\usepackage[table]{xcolor}
\usepackage{float}
\usepackage{lipsum}
\usepackage{tabularx}
\captionstyle{\center} %(para as legendas; use \legend para fonte)
\usepackage{caption}
\usepackage{hyperref}



% ---
% Pacotes de citações
% ---
% Paginas com as citações na bibl
\usepackage[brazilian,hyperpageref]{backref}
% Citações padrão ABNT
\usepackage[alf,abnt-emphasize=bf,abnt-etal-text=emph]{abntex2cite}  


% -- 
% Pacote para moldura
% --
% Adicionado para criar uma moldura ao redor da página
\usepackage{fancybox} 
% incluído a moldura da página
\fancypage{\setlength{\fboxsep}{12pt}\fbox}{} 
\usepackage{wrapfig}
%--
% comando para inserir autor e ano
\newcommand{\citeauthorandyear}[1]{\citeauthoronline{#1} (\citeyear{#1})}
\renewcommand{\ABNTEXchapterfontsize}{\bfseries\large}
% https://www.tablesgenerator.com/
\usepackage{float}
\floatstyle{plaintop} % Coloca caption no topo
\newfloat{quadro}{htbp}{lop}
\floatname{quadro}{Quadro}
\newcommand{\listofquadros}{\listof{quadro}{Lista de Quadros}}
\renewcommand{\baselinestretch}{1.5}

% --
% Autores
% --
\newcommand{\autorA}[1]{\def \oautorA {#1}}
\newcommand{\imprimirautorA}{\oautorA}
\newcommand{\prontuarioA}[1]{\def \oprontuarioA {#1}}
\newcommand{\imprimirprontuarioA}{\oprontuarioA}

\newcommand{\autorB}[1]{\def \oautorB {#1}}
\newcommand{\imprimirautorB}{\oautorB}
\newcommand{\prontuarioB}[1]{\def \oprontuarioB {#1}}
\newcommand{\imprimirprontuarioB}{\oprontuarioB}

\newcommand{\autorC}[1]{\def \oautorC {#1}}
\newcommand{\imprimirautorC}{\oautorC}
\newcommand{\prontuarioC}[1]{\def \oprontuarioC {#1}}
\newcommand{\imprimirprontuarioC}{\oprontuarioC}

\newcommand{\autorD}[1]{\def \oautorD {#1}}
\newcommand{\imprimirautorD}{\oautorD}
\newcommand{\prontuarioD}[1]{\def \oprontuarioD {#1}}
\newcommand{\imprimirprontuarioD}{\oprontuarioD}

\newcommand{\prof}[1]{\def \prof {#1}}
\newcommand{\imprimirprof}{\oprof}

\prof{Prof. Me. Rafael da Silva Muniz}
\titulo{Projeto de Extensão | Módulo 1 | 2023-01 }

% Não modificar - fim

% -- 
% -- 
% -- Atualizar a partir deste ponto
% -- 
% -- 


% -- Inserir sempre os nomes dos alunos em ordem alfabética
% ----- Caso não tenha mais alunos, basta inserir o símbolo % no início da linha  de \prontuarioB{}, \autorB{}, \prontuarioC{}, \autorC{}, \prontuarioD{}, \autorD{}

% Aluno A
\prontuarioA{BP9999999}
\autorA{Nome do autor A}
% Aluno B
\prontuarioB{BP9999999}
\autorB{Nome do autor B}
% Aluno C
\prontuarioC{BP9999999}
\autorC{Nome do autor C}
% Aluno D
\prontuarioD{BP9999999}
\autorD{Nome do autor D}
% ----

% -- 
% -- 
% -- Atualizar até aqui
% -- 
% -- 

% --
% Inicio do Documento
% --

% --
% -- Não modificar - inicio 
% --

\begin{document}
\selectlanguage{brazil}
%\thispagestyle{empty} % remove a numeração desta página
\textual

\begin{wrapfigure}[2]{l}{0.30\textwidth}
\includegraphics[scale=.35]{imagens/IFSP-BRA.png} 
\end{wrapfigure}

\centering
\footnotesize\textbf{Tecnologia em Análise e Desenvolvimento de Sistemas} \\	\footnotesize\textbf{Extensão como Metodologia de Ensino I - BRAEXT1} \\ 

\begin{center}
\vspace{1cm}
    \normalsize \textbf{\imprimirtitulo}
\end{center}

\begin{flushright}
    \prof
\end{flushright}

\begin{flushleft}
    \small \textbf{1. COMPOSIÇÃO DO GRUPO DE TRABALHO}
\end{flushleft}
\begin{center}
	\begin{tabular}{|m{0.5cm}|m{4cm}|m{10cm}|}
		\hline 
	    \textbf{n$^o$.} & \multicolumn{2}{c|}{\textbf{Prontuário | Nome do aluno}} \\
		\hline
		\textbf{1} & \textit{\imprimirprontuarioA} & \textit{\imprimirautorA}\\
		\hline
		\textbf{2} & \textit{\imprimirprontuarioB} & \textit{\imprimirautorB}\\
		\hline
		\textbf{3} & \textit{\imprimirprontuarioC} & \textit{\imprimirautorC}\\
		\hline
        \textbf{4} & \textit{\imprimirprontuarioD} & \textit{\imprimirautorD}\\
		\hline
	\end{tabular}
\end{center}

\begin{flushleft}
   \normalsize\textbf{2.TÍTULO E PROPOSTA DO PROJETO}
\end{flushleft}

\begin{flushleft}
   \small \textbf{Tema do Projeto:}
\end{flushleft} 
\normalsize
% -- 
% --- Alterar o Título do Projeto, se necessário:
Portal de demandas externas\\


\begin{flushleft}
\textbf{Elementos obrigatórios: }
\end{flushleft} 
% --- Para inserir o problema de pesquisa, localize o arquivo e altere
\justify

\begin{enumerate}
    \item \textbf{Logo}. \\Criação de uma logo para o Portal de Demandas da comunidade externa.  
    \item \textbf{Menu} com as opções: \\ Home, Instruções, Demandas, Sobre, Contato.
    \item \textbf{Home}. \\A página inicial deve conter: Título do portal com logo e uma imagem de fundo que represente o IFSP-BRA e um texto descritivo.
    \item \textbf{Instruções}. \\Página com instruções de como solicitar a demanda e como consultar.
    \item \textbf{Demandas}. \\Serão duas partes: uma contendo um Formulário com as informações necessárias para enviar uma demanda ou Link para um formulário do google (criação do formulário no google). E outra, com a listagem de Demandas e suas respectivas situações: solicitadas, em análise, em andamento, atendida.
    \item \textbf{Contato}. \\Dados de contato como email, telefone. Um formulário de contato direto (nome, email, telefone, mensagem).
    \item \textbf{Sobre}. \\Página com texto apresentando o que é a curricularização da extensão, links, imagens etc. 
    \item \textbf{Rodapé}. \\Inserir um mapa da localização do IFSP-BRA, endereço, telefones, redes sociais, e o nome dos integrantes da equipe que elaborou o site.
\end{enumerate}


\vspace{1cm}

\begin{center}
	\begin{tabular}{|m{10cm}|m{5cm}|}
		\hline
		\textbf{Alunos} & \textbf{Assinatura} \\
		\hline
		 \textit{\imprimirprontuarioA - \imprimirautorA} &  \\
		\hline
		\textit{\imprimirprontuarioB - \imprimirautorB} &  \\
		\hline
		\textit{\imprimirprontuarioC - \imprimirautorC} & \\
		\hline
         \textit{\imprimirprontuarioD - \imprimirautorD} & \\
		\hline
	\end{tabular}
\end{center}

\begin{flushright}
   Data: \today.
\end{flushright}

\end{document}
% --
% -- Não modificar - fim
% --

